\documentclass[12pt,a4paper]{book}
\usepackage[left=25mm, top=20mm, right=25mm, bottom=20mm, nofoot]{geometry}
\usepackage[utf8]{inputenc}
\usepackage[russian]{babel}
\usepackage[OT1]{fontenc}
\usepackage{amsmath}
\usepackage{amsfonts}
\usepackage{amssymb}
\usepackage{graphicx}
\author{Denis Korzhenkov}
\title{Make E-commerce A/B Testing Great Again}
\begin{document}
\section{Предисловие}
Сегодня значительную роль в маркетинговых исследованиях играет так называемое A/B тестирование. Вкратце его суть можно описать следующим образом: вы пытаетесь решить, красной или синей сделать кнопку "Купить!" на своем сайте. Поскольку углублять в нейрофизиологические тонкости не хочется, вы решате просто разделить всех приходящих на сайт на две группы, и половине показать первую кнопку, а другой половине -- вторую. При этом важно, чтобы при повторном заходе эти половины (т.н. "вариации") не перемешивались, что решается, к примеру, проставлением cookies в браузере посетителя. Вы продолжаете тест какое-то время, затем подсчитываете для каждой вариации общее количество посетителей, подсчитываете, сколько из них нажало на свой вариант кнопки, делите одно на другое (получая "конверсию") и решаете, какой же вариант победил. \\
В этой благостной картине сразу же рождаются вопросы. Сколько ждать? Как сравнить? Обязательно ли должен быть победитель? В Сети немало руководств в стиле "делай как я," но при этом достаточного обоснования для именно такого порядка действий почти не приводится. Там же, где попытки есть, часто встречаются ошибки, неявные умолчания и другие неприятные вещи.\\
Автор при подготовке этого пособия главной целью ставил собственное ознакомление с имеющимися методами. При чтении рекомендуется иметь представление о теории вероятности и математической статистике.
\subsection{Базовые понятия теории вероятностей}
Мы опустим технические тонкости, постаравшись сделать это не в ущерб изложению материала.\\
\subsubsection{Функция распределения и плотность}
Рассмотрим случайную величину $X = \left(X_1, \dots, X_n \right)$, принимающую значения на $n$-мерном множестве вещественных чисел $\mathbb{R}^n$. Будем считать, что для любого подмножества $A \subset \mathbb{R}^n$ определена вероятность того события, что $X$ попала в $A$: $$\Pr\left(X \in A \right).$$
Назовем функцией распределения $F\left(x_1, \dots, x_n\right)$ вероятность того, что $\forall i$ компонента случайной величины $X_i$ не превосходит соответствующего аргумента функции $F$:
$$ F\left(x_1, \dots, x_n\right) = \Pr\left(X_1 \leq x_1, \dots, X_n \leq x_n\right).$$

Плотностью же величины $X$ принято называть такую функцию $f \left(x_1, \dots, x_n\right),$ что для произвольного подмножества $A \subset \mathbb{R}^n$ верно
$$ \Pr\left(X \in A \right) = \int_A f \left(x_1, \dots, x_n\right) dx_1 \dots dx_n.$$
Но для наших целей понятие плотности можно упростить. Во-первых, разделим случайные величины на дискретные и непрерывные. Если, чтобы занумеровать все возможные значения $X$, нам хватит натуральных чисел, будем называть ее дискретной. Если же ее значения заполняют какую-то часть пространства "без промежутков" (возможно, значения заметают не все $\mathbb{R}^n$), то будем называть ее непрерывной. Конечно, эти определения не особенно строги, но для общего понимания этого достаточно.\\
В случае дискретной случайной величины будем называть плотностью в конкретной точке просто вероятность того, что случайная величина приняла эту точку в качестве значения. Очевидно, в этом случае плотность отлична от нуля толко в тех точках, куда может "попасть" случайная величина. "Интегральное" определение плотности при этом сохраняется, только интеграл при этом надо рассматривать, как говорят, по считающей мере -- представляя его как сумму по всем возможным точкам с ненулевой плотностью.\\
Для непрерывной же величины предположим, что ее функция распределения дифференцируема, и определим плотность как
$$ f \left(x_1, \dots, x_n\right) = \dfrac{\partial^n F\left(x_1, \dots, x_n\right)}{\partial x_1 \dots \partial x_n}.$$
\subsubsection{Математическое ожидание и дисперсия}
Перейдем к рассмотрения одномерных случайных величин. Для непрерывной случайной величины $X$ с плотностью $f \left( x \right)$ определим математичское ожидание как 
$$ \mathbb{E}X = \int_\mathbb{R} x f\left( x \right) dx.$$
Если переделать это в интеграл по считающей мере, получим определение математического ожидания для дискретных величин
$$ \mathbb{E}X = \sum_i x_i \Pr \left(X = x_i \right), $$
где $\{ x_i \}_i$ - множество всех значений, принимаемых дискретной случайной величиной $X$. Как видно (особенно на примере дискретных величин), математическое ожидание есть некое среднее значение величины. Поэтому часто его называют просто средним.\\

Дисперсией случайной величины $X$ называют математическое ожидание квадрата ее отклонения от среднего:
$$ \mathbb{D}X = \mathbb{E} \left( X - \mathbb{E}X \right)^2 = \mathbb{E} X^2 - \left( \mathbb{E}X \right)^2. $$
(В англоязычной литературе принято обозначение $Var \, X.$) Интуитивно ясно, что дисперсия как-то характеризует разброс значений случайной величины относительно среднего.\\

Корень квадратный из дисперсии называется стандартным отклонением величины $X$.
\subsubsection{Независимость и условные распределения}
Рассмотрим два подмножества $A$ и $B$, лежащие в $\mathbb{R}$ (впрочем, для многомерного случая все сказанное остается верным без малейшего изменения в рассмотрении), и две одномерные случайные величины $X$ и $Y$. Скажем, что события $X \in A$ и $Y \in B$ независимы, если выполнено равенство
$$ \Pr \left( X \in A, \, Y \in B \right) = \Pr \left( X \in A \right) \cdot \Pr \left( Y \in B \right), $$
то есть вероятность одновременного выполнения двух событий распадается в произведение соответствующих "одиночных" вероятностей.\\

Случайные величины $X$ и $Y$ называются независимыми, если для любых подмножеств $A$ и $B$ события $X \in A$ и $Y \in B$ независимы. Неформально говоря, $X$ и $Y$ "не интересуются" поведением друг друга.
\subsubsection{Некоторые распределения}
Рассмотрим 

\end{document}